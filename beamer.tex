\catcode`~=11 % make LaTeX treat tilde (~) like a normal character
\newcommand{\urltilde}{\kern -.15em\lower .7ex\hbox{~}\kern .04em}
\catcode`~=13 % revert back to treating tilde (~) as an active character
\providecommand{\abs}[1]{\left | #1\right |}

\documentclass{beamer}
\usepackage[latin1]{inputenc}
\usetheme{default}
%Warsaw
\title[The Electronic Properties Of Graphene]{The Electronic Properties Of Graphene:\\Theoretical Modelling of Doping}
\author{James McMurray}
\institute{Supervisor: Dr. Shytov\\Quantum Systems and Nanomaterials Group\\University Of Exeter}
\date{11/11/2011}
\begin{document}


\pgfdeclareimage[height=0.5cm]{logo}{exlogo.png}
\logo{\pgfuseimage{logo}}

\begin{frame}
\titlepage
\end{frame}

  \begin{frame}<beamer>
    \frametitle{Outline of talk}
    \tableofcontents
  \end{frame}

\section{Introduction}
\subsection{What is Graphene?}
\begin{frame}[t]{What is Graphene?}
\begin{itemize}
\item Formed from Carbon atoms in a hexagonal lattice. 
\item First 2D crystal observed experimentally - Geim and Novoselov 2004.
\item Produced by the micromechanical exfoliation of graphite:\\
\begin{enumerate}
\item Use sticky tape to repeatedly peel off thin layers of carbon.
\item Dissolve the sticky tape in solution.
\item Transfer layers to a Silicon Dioxide substrate.
\item Use an optical microscope to observe interference from Graphene samples.
\end{enumerate}
\end{itemize}

\begin{figure}[c]
\includegraphics[height=2cm]{graphenediag.png}
\caption{A diagram of the basic structure of Graphene. (Taken from \url{http://www.chem1.com/acad/webtext/states/states.html})}
\end{figure}

\end{frame}

%What are quasiparticles
%What defects?


\subsection{Possible Applications of Graphene}
\begin{frame}[t]{Possible applications of Graphene}
\begin{itemize}
\item Possible uses in high frequency transistors
\begin{itemize}
\item  But Graphene Field Effect Transistors show low on-off ratio - ``off" state still has relatively high conduction compared to ``on" state
\item Large energy wastage
\item Need band gap
\item Can be engineered through doping - the addition of Hydrogen to produce ``graphane" or fluorine to produce fluorinated graphene.
\end{itemize}
\end{itemize}

\begin{figure}[c]
\includegraphics[height=2.5cm]{graphanecrys2.jpg}
\caption{An artist's impression of graphane, Carbon atoms in blue, Hydrogen atoms in red. Taken from \url{http://www.physorg.com/news152545648.html}}
\end{figure}
\end{frame}


\section{Our Aim}
\begin{frame}[t]{The Aim of our project}
\begin{itemize}
\item Computationally model the effect of defects in graphene
\item Relevant for Hydrogenated and Fluorinated graphene
\item Basic methodology:
\begin{enumerate}
\item Compute the Hamiltonian to obtain the density of states and eigenvectors for the system - using the Tight Binding Model
\item Use this to calculate the transmission coefficients - considering the transmission at a potential barrier
\item Use these to calculate the conductance of the modelled sample - using the Landauer formula
\end{enumerate}
\item Will also later model the effect of magnetic fields to investigate the  Quantum Hall Effect in graphene (not included in this talk)
\end{itemize}
\end{frame}

\section{Theory of Graphene}
\subsection{Crystal Lattice structure}
\begin{frame}[t]{Crystal Lattice structure - Real Lattice}
\begin{itemize}
\item Carbon atoms in a hexagonal lattice
\item Two unique sublattices labelled A and B.
\item Lattice vectors: $\mathbf{a_{1,2}} = \frac{3a}{2} ~\mathbf{\hat i}~ \pm~ \frac{\sqrt{3}a}{2} ~\mathbf{\hat j}$
\item Lattice constant: $a \approx 2.461\text{\AA}$
\end{itemize}

\begin{figure}[c]
\includegraphics[height=4cm]{mygraphenediag.png}
\caption{A diagram of the real space lattice of graphene.}
\end{figure}
\end{frame}

\subsection{The Tight Binding Model}
\begin{frame}[t]{The Tight Binding Model - Calculating the Hamiltonian}
\begin{itemize}
\item Assume electrons reside on lattice sites only
\item Assume that electrons can only hop to nearest neighbours (can include higher nearest neighbour interactions)
\item So electrons can only move from sublattice A points to sublattice B points and vice versa.
\item In the Hamiltonian matrix, $H_{ij}$ is the energy cost for hopping from position $i$ to position $j$.
\end{itemize}
\begin{figure}[c]
\includegraphics[height=3cm]{hopping.png}
\caption{A diagram showing hopping possibilities from a Carbon atom in the lattice.}
\end{figure}
\end{frame}

\begin{frame}[t]{The Tight Binding Model (continued)}

\begin{figure}[c]
\includegraphics[height=3cm]{hopping2.png}
\caption{Labelling the sites 1-4.}
\end{figure}
\begin{itemize}
\item With a nearest-neighbour hopping energy $t$ ($t \approx 2.8eV$), the Hamiltonian is:
\begin{equation}
\hat {\text{H}} = \left ( \begin{array}{cccc}
E_0 & -t & -t & -t\\
-t & E_1 & 0 & 0\\
-t & 0 & E_2 & 0\\
-t & 0 & 0 & E_3\\
\end{array}
\right )
\end{equation}
\end{itemize}
\end{frame}

\begin{frame}[t]{The Tight Binding Model (continued)}
\begin{itemize}
\item By considering the lattice geometry, and the sum of the wavefunctions across both sublattices, the following expression is obtained for the Time Independent Schr\"{o}dinger Equation:
\begin{equation}
E \psi = \left ( \begin{array}{cc}
0 & -\tilde t(k)\\
-\tilde t^* (k) & 0\\
\end{array} \right ) \psi
\end{equation}
where
\begin{equation}
\tilde{t(k)}=t\left ( e^{ik_x a} + e^{\frac{-ik_x a}{2}} 2\cos\left(\frac{k_y \sqrt{3}a}{2} \right ) \right )
\end{equation}
\item Taking the eigenvalues of the matrix yields:
\begin{equation}
E = \pm \abs{\tilde t(k)}
\end{equation}
\end{itemize}
\end{frame}

\begin{frame}[t]{Dirac Points on band structure}
\begin{figure}
\includegraphics[height=6cm]{defecttransport.png}
\caption{A diagram of the band structure of graphene. The Dirac points are where the bands converge. Taken from \url{http://nextbigfuture.com/2011/04/nrl-researchers-take-step-toward.html}}
\end{figure}
\end{frame}

\subsection{Dirac Points}
\begin{frame}[t]{Dirac Points}
\begin{itemize}
\item Points in reciprocal space where  $\abs{(\tilde t(k))} =0$.
\item These lie on the vertices of the first Brillouin Zone in reciprocal space.
\item The unique Dirac points are K and K'.
\end{itemize}
\begin{figure}[c]
\includegraphics[height=4cm]{kspacediag.png}
\caption{A diagram of the first Brillouin zone of graphene in reciprocal space.}
\end{figure}
\end{frame}

\begin{frame}[t]{Dispersion near the Dirac points}
\begin{itemize}
\item Near the Dirac points the Hamiltonian reduces to:
\begin{equation} \hat{H} = \hbar v \left ( \begin{array}{cc}
0 & k_x - ik_y \\
k_x + ik_y & 0\\
\end{array} \right )
\end{equation}
\item Taking the eigenvalues of this yields:
\begin{equation}
E = \pm \hbar v k
\end{equation}
\item This is a linear dispersion relation, and means that the quasiparticles are massless fermions near the Dirac points - they act as relativistic particles with a constant velocity $v \approx 10^6 \text { m s}^{-1}$
\item Note that since the Dirac points are points in \emph{reciprocal} space, this relation applies for electron transport through-out the graphene sample as energy is always small compared to peak energy ($t\approx 2.8$ eV corresponds to $\urltilde$30,000 K).
\end{itemize}
\end{frame}

\begin{frame}[t]{Showing the effective mass is zero at the Dirac points}
\begin{itemize}
\item At the Dirac points we obtained the dispersion relation:
\begin{equation}
E=\hbar v k
\end{equation}
\item From Special Relativity we know:
\begin{equation}
E=\sqrt{p^2c^2 + m^2c^4}
\end{equation}
\item So the dispersion relation is just the $m\rightarrow0$ case of this.
\begin{equation}
\therefore E=\sqrt{p^2c^2} = pc = \hbar k c
\end{equation}
\item Constant velocity is $v\approx 10^6 \text{ m s}^{-1}$  instead of c ($\urltilde 3\times 10^8 \text{ m s}^{-1}$).
\item So effective mass is zero and relativistic effects much more apparent.
\end{itemize}
\end{frame}

\begin{frame}[t]{Dispersion near the Dirac points}
\begin{itemize}
\item This means that near the Dirac points, the dispersion follows ``Dirac cones":
\begin{figure}[c]
\includegraphics[height=4cm]{honeylattice.png}
\caption{A diagram showing Dirac Cones at the Dirac points in reciprocal space. Taken from:  \url{http://www.mpipks-dresden.mpg.de/mpi-doc/CondensedMatter//content/QuantHall.shtml}}
\end{figure}
\end{itemize}


\end{frame}

\subsection{Klein Paradox}
\begin{frame}[t]{Klein Paradox}
\begin{itemize}
\item The Klein Paradox results as a consequence of the massless Dirac fermion quasiparticles, and their linear dispersion.
\item Unintuitive result that transmission probability increases with the height of the potential barrier, and tends to 1 as the potential tends to infinity.
\item This means that the quasiparticles cannot be localised.
\item We should observe this effect when calculating transmission coefficients.
\end{itemize}
\begin{figure}[c]
\includegraphics[height=3cm]{diraccones.png}
\caption{A diagram showing the principle of how the transmission occurs by shifting the quasiparticle to the lower cone.}
\end{figure}
\end{frame}

\subsection{The Landauer Formula}
\begin{frame}[t]{The Landauer Formula - calculating the conductance}
\begin{itemize}
\item By considering the electrons that are able to move across a potential and the density of states of graphene near the Dirac points, the following equation is obtained for the net conductance:
\begin{equation}
G \approx \frac{e^2}{\pi \hbar} \sum_i T_i
\end{equation}
where the sum is taken across all channels (taking in to account the possibility for varying transmission amplitudes)
\item So per channel the conductivity is:
\begin{equation}
\sigma \approx \frac{e^2}{\pi \hbar}
\end{equation}
\item Note that this means there is a minimum conductivity even in ideal, undoped Graphene, where there should be no scattering and no net current carriers!
\end{itemize}
\end{frame}

\subsection{The Missing Pi Problem}
\begin{frame}[t]{The Missing Pi Problem}
\begin{itemize}
\item But experimental results show the minimum conductivity to be an order of $\pi$ larger:
\begin{equation}
\sigma_{\text{experimental}} \approx \frac{e^2}{\hbar}
\end{equation}
\item This is known as the ``Missing Pi(e) Problem"
\item Problem still unsolved
\item But 2007 report by Miao \textit{et al.} in Science \textbf{317} 3150, reported that the experimental value approached the theoretical value for specific shapes - small, wide graphene rectangles were closer to theory.
\end{itemize}
\end{frame}


\subsection{Summary}
\begin{frame}[t]{Summary: The interesting theory of Graphene}
\begin{itemize}
\item Exhibits high crystal quality - electrons can thousands of interatomic distances without scattering.
\item Massless Dirac fermions near Dirac points - show relativistic effects in ordinary conditions!
\item Klein Paradox - Particles tunnel through a potential barrier with a transmission coefficient which tends to 1 as the potential tends to infinity.
\item Universal minimum conductivity:
\begin{itemize}
\item Finite conductivity in undoped graphene at Dirac points (where there are no net charge carriers) even at low T (when there is no scattering)
\item Conductivity without charge carriers!
\item Normally materials would transition to insulators at low T
\item But in Graphene there is a suppression of localisation
\end{itemize}
\item Missing Pi problem - observed minimum conductivity is a multiple of pi greater than predicted by theory
\end{itemize}
\end{frame}

\section{Preliminary Results}

\begin{frame}[t]{Preliminary Work}
\begin{itemize}
\item We have already started work on the project
\item Program written in C
\item Produce Hamiltonian eigenvectors and eigenvalues
\item Produce density of states
\item Already have data for the density of states plotted with a fit from an analytical approach by Dr. Shytov
\item Resolution of histogram limited by the size of the modelled sample
\end{itemize}
\end{frame}

{
\usebackgroundtemplate{\includegraphics[width=\paperwidth]{shytovfit450.png}}
\begin{frame}[plain]
\end{frame}
}

\begin{frame}[t]{Conclusion - What is to be done?}
\begin{itemize}
\item Add method to normalise the density of states per unit area
\item Add method for modelling defects
\begin{itemize}
\item For vacancies this means setting the energy to occupy a point to a very large number
\end{itemize}
\item Add methods to obtain transmission coefficients and conductance
\item Will later consider magnetic field effects
\item Some computational challenges due to the size of the considered samples
\begin{itemize}
\item High memory usage if all zeros are stored as double precision floats too
\end{itemize}
\item The Hamiltonian is sparse and so we will attempt to use sparse matrix methods
\end{itemize}
\end{frame}

\begin{frame}[t]{Questions?}
\begin{center}
{\huge Any questions?}
\end{center}
\begin{itemize}
\item Thank you for your time!
\item Thanks to my supervisor Dr. Shytov for his help and discussions.
\item Thanks to my colleagues Chris Beckerleg and Will Smith for their continuing work.
\item Presentation produced using \LaTeX ~ and Beamer.
\end{itemize}
\end{frame}

\begin{frame}{Supplementary slides}
\begin{center}
{\Huge Supplementary slides}
\end{center}
\end{frame}

\begin{frame}[t]{Why is the ``impossible" possible?}
\begin{itemize}
\item From the Mermin-Wagner theorem, thermal fluctuations would	 make 2D structures unstable at any finite temperature.
\item This was confirmed as the melting temperature of thin films rapidly decreased with decreasing temperature\footnote{See Geim \& Novoselov, The rise of graphene, \emph{Nat. Mat.} {\bf 6}, 183 (2007).}.
\item In graphene, the gentle 3D warping (\urltilde10nm) provides stability by minimising thermal vibrations.
\begin{figure}[c]
\includegraphics[height=3cm]{corrugatedgraphene.jpg}
\caption{An artist's impression of corrugated Graphene: Jannik Meyer. (Taken from \url{http://www.nanowerk.com/})}
\end{figure}
\end{itemize}
\end{frame}



\begin{frame}[t]{Why is localisation suppressed?}
\begin{itemize}
\item The suppression of localisation is a result of the relativistic effects on the quasiparticles.
\item The quasiparticles cannot be localised due to the Zitterbewegung (jittery motion) effect.
\item The Klein effect also means that the quasiparticles are transmitted through potential barriers with a probability $\approx 1$.
\item This is also contributes to the suppression of localisation.
\end{itemize}
\end{frame}

\begin{frame}[t]{Translation to ``brick wall" model for computation}
\begin{figure}[c]
\includegraphics[height=3cm]{translation.png}
\caption{Diagram showing the principle of the translation.}
\end{figure}
\begin{itemize}
\item We translate to the "brick-wall" model when considering the bonds.
\item So all bonds are straight horizontal or vertical - no $\sqrt{3}$ factors.
\item But we must reconsider these when normalising the density of states per unit area and other considerations.
\end{itemize}
\end{frame}

\begin{frame}[t]{The Tight Binding Approximation derivation}
\begin{itemize}
\item The Time Independent Schr\"{o}dinger Equation (TISE) states:
\begin{equation}
E\psi = \hat{H} \psi 
\end{equation}
\item From Bloch theory we know:
\begin{equation}
\psi_A = \tilde \psi_A e^{i(\mathbf{k} \cdot \mathbf{r})}, ~ \psi_B = \tilde \psi_B e^{i(\mathbf{k} \cdot \mathbf{r})}
\end{equation}
\item So from point 1, the TISE becomes:
\begin{equation}
E \tilde\psi_A = - \sum_{i,j}H_{ij}\psi_{j} = H_{12}\psi_2 + H_{13}\psi_3 + H_{14}\psi_4 = -t(\psi_2 + \psi_3 + \psi_4)
\end{equation}
\item But recall that:
\begin{equation}
\mathbf{k} \cdot \mathbf{r} = k_x x + k_y y
\end{equation}
\end{itemize}
\end{frame}

\begin{frame}[t]{The Tight Binding Approximation derivation (2)}
\begin{itemize}
\item So by considering the geometry of the lattice points, assuming $\mathbf r$ has its origin at point 1, the TISE becomes:
\begin{equation}
E \tilde \psi_A = -t \left ( \tilde \psi_B \left ( e^{ik_x a} + e^{\frac{-ik_x a}{2} + \frac{i k_y a \sqrt{3}}{2}} + e^{\frac{-i k_x a}{2} - \frac{i k_y a \sqrt{3}}{2}} \right ) \right )
\end{equation}
\item Which can be written as:
\begin{equation}
E \tilde \psi_A = -t(k) \tilde \psi_B
\end{equation}
\item Combining this with an equation for a point on sublattice B (repeating the same steps), one obtains:
\begin{equation}
E \psi = \left ( \begin{array}{cc}
0 & -\tilde t(k)\\
-\tilde t^* (k) & 0\\
\end{array} \right ) \psi
\end{equation}
\end{itemize}
\end{frame}

\begin{frame}[t]{Derivation of Hamiltonian near Dirac points}
\begin{itemize}
\item From previous derivation:
\begin{eqnarray*}
\abs{\tilde t(k)} &=& t\abs{ e^{ik_x a} + e^{\frac{-ik_x a}{2} + \frac{i k_y a \sqrt{3}}{2}} + e^{\frac{-i k_x a}{2} - \frac{i k_y a \sqrt{3}}{2}}} \\&= &t\abs{ e^{ik_x a} + e^{\frac{-ik_x a}{2}} 2\cos\left(\frac{k_y \sqrt{3}a}{2} \right )} =0
 \end{eqnarray*}
 Near K and K'.
 \item Dividing by $e^{ik_x a}$:
 \begin{equation}
 \abs{\tilde t(k)}= t\abs{ 1 + e^{\frac{-i3k_x a}{2}} 2\cos\left(\frac{k_y \sqrt{3}a}{2} \right )} 
 \end{equation}
 \item Using Taylor approximation near K:
 \begin{equation}
 \tilde t(K+\delta k) \approx \tilde t(K) + \delta k_x \frac{\partial \tilde t(k)}{\partial k_x} + \delta k_y \frac{\partial \tilde t(k)}{\partial k_y}
 \end{equation}
\end{itemize}
\end{frame}

\begin{frame}[t]{Derivation of Hamiltonian near Dirac points (2)}
\begin{itemize}
\item Calculating the differentials:
\begin{eqnarray*}
 \tilde t(K+\delta k) \approx \tilde t(K) &+& \delta k_x \left ( \frac{-3ai}{2} e^{\frac{-ik_x 3a}{2}} \cdot 2 \cos \left (\frac{k_y \sqrt{3}a}{2} \right ) \right )t\\& + &\delta k_y \left (-\sqrt{3} a \sin \left (\frac{k_y a \sqrt{3}}{2} \right ) e^{\frac{-ik_x 3a}{2}} \right )t 
 \end{eqnarray*}
 \item Substituting for K: $\left (0, \frac{4\pi}{3\sqrt{3}a} \right)$
 \begin{equation}
  \tilde t(K+\delta k) \approx 0 + \delta k_x \left ( \frac{3ait}{2} \right ) + \delta k_y \left (\frac{-3at}{2} \right )
  \end{equation}
  \item So the Hamiltonian can be written:
  \begin{equation}
  H_K (k) = \frac{3at}{2}\left ( \begin{array}{cc}
  0 & k_y - i k_x\\ 
  k_y + ik_x & 0\\ \end{array} \right )
  \end{equation}
\end{itemize}
\end{frame}


\begin{frame}[t]{Derivation of Hamiltonian near Dirac points (3)}
\begin{itemize}
\item Can write this in terms of Pauli matrices $\sigma$:
\begin{equation}
\sigma_x = \left ( \begin{array}{cc} 0 & 1\\ 1 & 0\\ \end{array} \right ) ~,~ \sigma_y = \left ( \begin{array}{cc} 0 & -i\\ i & 0\\ \end{array} \right )
\end{equation}
\item Bust must swap $k_x$ and $k_y$ to make this simple (rotating the k vector):
\begin{equation}
k_x \rightarrow k_y ~,~ k_y \rightarrow -k_x
\end{equation}
\item So the Hamiltonian at K becomes:
  \begin{equation}
  H_K (k) = \frac{3at}{2}\left ( \begin{array}{cc}
  0 & -k_x - i k_y\\ 
  -k_x + ik_y & 0\\ \end{array} \right )
  \end{equation}
  \item Which can be written as:
  \begin{equation}
  H_K(k) = -\frac{3at}{2}\left ( \vec{\sigma^*} \cdot \vec{k} \right )
  \end{equation}
\end{itemize}
\end{frame}

\begin{frame}[t]{Derivation of Hamiltonian near Dirac points (4)}
\begin{itemize}
\item Similarly for K': $\left (0, \frac{-4\pi}{3\sqrt{3}a} \right)$
\begin{equation}
  H_{K'}(k) = \frac{3at}{2}\left ( \vec{\sigma} \cdot \vec{k} \right )
  \end{equation}
    \item Taking the eigenvalues of $H_K(k)$:
  \begin{equation}
  \left | \begin{array}{cc} E & \frac{3at}{2} \left (  -k_x - i k_y \right ) \\ \frac{3at}{2} \left (  -k_x + ik_y \right ) & E\\ \end{array} \right | =0
  \end{equation}
    \begin{eqnarray}
  &\therefore& E^2 - \left ( \frac{3at}{2} \right )^2 \left (k_x^2 - ik_x k_y + i k_x k_y + k_y ^2 \right )\\
  &\therefore& E^2 - \left ( \frac{3at}{2} \right )^2 k^2 =0 ~ \therefore E = \pm \frac{3at}{2} k
  \end{eqnarray}
\item Linear dispersion - can be written as $E = v\hbar k$
\end{itemize}
\end{frame}



%Why trigonal warping
%Derivation of Landauer formula
%Klein Paradox formula
%Van Hove singularity
%What is a field effect transistor?

\end{document}