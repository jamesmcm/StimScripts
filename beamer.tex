%\catcode`~=11 % make LaTeX treat tilde (~) like a normal character
%\newcommand{\urltilde}{\kern -.15em\lower .7ex\hbox{~}\kern .04em}
%\catcode`~=13 % revert back to treating tilde (~) as an active character
%\providecommand{\abs}[1]{\left | #1\right |}

\documentclass{beamer}
\usepackage[latin1]{inputenc}
\usetheme{default}
%Warsaw
\title[Summary of work]{Summary of work}
\author{James McMurray}
\institute{Supervisor: Nora Umbach\\Department of Psychology\\Eberhard-Karls Universit\"{a}t T\"{u}bingen}
\date{30/08/2012}
\begin{document}


\pgfdeclareimage[height=0.5cm]{logo}{exlogo.png}
\logo{\pgfuseimage{logo}}

\begin{frame}
\titlepage
\end{frame}

  \begin{frame}<beamer>
    \frametitle{Outline of talk}
    \tableofcontents
  \end{frame}


\section{TU Berlin Code}
\begin{frame}[t]{TU Berlin Code}
Provided group of testing scripts for CRT/LCD monitor, and also scripts for producing various stimuli. {\bf Calibration Tests :}
\begin{description}
\item[crttest.py] Produces a patch of a certain luminance in the centre of the screen, and modulates the surround according to a sin function. On a CRT the luminance of the patch will vary, but not on an LCD. Included as {\it CRTTest()} in {\it stimuliclass.py}.
\item[gamma.py] Compare luminance of large patch and small patch, on a CRT, the large patch should always be less bright. Can use up and down arrow keys to modify the size of the patch whilst running. Included as {\it PatchBrightnessTest()} in {\it stimuliclass.py}
\item[lines.py] Create series of lines, swap between horizontal and vertical - check for change in overall luminance. Included as {\it Lines()} in {\it stimuliclass.py} 
\item[singrating.py] Produces a sin-wave based stimuli across the screen, which is then shifted to anti-phase and re-presented rapidly. Apparently should be invisible at high frequencies. Included as {\it SinGrating()} in {\it stimuliclass.py}

\end{description}
\end{frame}
\begin{frame}[t]{Stimuli}
Also provided methods to produce the following stimuli. Now implemented in {\it stimuliclass.py}.
\begin{description}
\item[Cornsweet()] Produces a form of the Cornsweet illusion to PNG if no PNG file is provided in the pngfile argument, otherwise it will display the stimuli provided.
\item[Mondrian()] Produces a Mondrian to a PNG if no PNG file is provided in the pngfile argument, otherwise it will display the Mondrian with run(). This method is called during the generation of articulated stimuli.
\item[Todorovic()] Produces a form of the Torodovic checkerboard illusion to PNG if no PNG file is provided in the pngfile argument, otherwise it will display the stimuli provided. Note that it works by first producing an appropriate Cornsweet stimulus and then repeating this.
\item[WhiteIllusion()] Produces a form of the White's illusion on a square wave to PNG if no PNG file is provided in the pngfile argument, otherwise it will display the stimuli provided. Produces both kind=``bmcc'': in the style used by Blakeslee and McCourt (1999), and kind=``gil'': in the style used by Gilchrist (2006).

%Perhaps split these on different frames with images
\end{description}

\end{frame}

\section{Articulated Stimuli}


\begin{frame}[t]{Articulated Stimuli}
%photo here

The stimuli are produced via the {\it articulated.py} script in {\bf achrolabutils}. 

\begin{itemize}
\item This script uses the \emph{normal distribution} to set the shades of gray, so one provides the \alert{mean} and \alert{standard deviation} of the values for both sides along with a \alert{seed} to randomise the pattern, and the \alert{mean edge length of the Mondrian rectangles}.

\item In the {\it articulated.py} script one must change the stimuli list (of infield and surround values). The script then produces all of the stimuli, matched against eachother, encoded for the eizo monitor. Currently it produces them both with transparent infields, and with non-transparent infields.
\end{itemize}
\end{frame}


%Problems - calibration beanplots, lines problem
\end{document}
